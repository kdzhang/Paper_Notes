\documentclass{book}
% \usepackage[utf8]{inputenc}

\usepackage{amssymb,amsmath,amsfonts,eurosym,geometry,graphicx,caption,color,setspace,sectsty,footmisc,caption,array,hyperref,dsfont}

% \usepackage{indentfirst}	% make indent in first paragraph
\setlength\parindent{0pt}
\setlength{\parskip}{1em}
% \renewcommand{\baselinestretch}{1.5}

\usepackage{bm}
\usepackage{geometry}		% to make typesettings like margins
\usepackage{mathrsfs}

% \usepackage{natbib}			% to insert citations
\usepackage[natbib,authordate,noibid,backend=biber]{biblatex-chicago}
\addbibresource{/Users/mac/Dropbox/Academic_Writings/reference-macwin.bib}
% \addbibresource{C:/Users/Kaida/Dropbox/Academic_Writings/reference-win.bib}


\usepackage[table,xcdraw]{xcolor}			% extend the colors can be used

% \usepackage{bibentry}		% package used to insert full citation in body text
% replaced by \fullcite in biblatex

% \makeatletter 
% \renewcommand\BR@b@bibitem[2][]{\BR@bibitem[#1]{#2}\BR@c@bibitem{#2}}
% \makeatother
% \nobibliography*


\usepackage{setspace}		% set space for comment bullet

\usepackage{graphicx}		% to insert image
\graphicspath{ {images/} }

\usepackage[english]{babel}

\usepackage{tocbibind}		% to add toc and bib into table of content
\usepackage{bookmark}	% to separate the last chapter from previous ones

\usepackage{amsthm}
\makeatletter
\def\th@plain{%
  \thm@notefont{}% same as heading font
  \itshape % body font
}
\def\th@definition{%
  \thm@notefont{}% same as heading font
  \normalfont % body font
}


\theoremstyle{plain}
\newtheorem{thm}{Theorem}[section] % reset theorem numbering for each chapter
\theoremstyle{definition}
\newtheorem{defn}{Definition}[section] % definition numbers are dependent on theorem numbers
\newtheorem{exmp}{Example}[section] % same for example numbers
\newtheorem{lemma}[thm]{Lemma}
\newtheorem{prop}[thm]{Proposition}
\newtheorem{obs}{Observation}
\newtheorem{que}{Question}[section]
\newtheorem{assp}{Assumption}[section]

\newenvironment{myproof}
{\noindent\textit{Proof:}}{\hfill$\square$}

\newenvironment{answer}
{\noindent\textit{Answer:}}{\hfill$\square$}

% \citestyle{chicago}
\geometry{letterpaper, margin=0.75in}
% \setlength\parindent{0pt} % to cancel indent


% A list of new commands
\newcommand{\R}{\mathbb{R}}			% depends on the package amssymb
\newcommand{\F}{\mathcal{F}}
\newcommand{\myline}{\vspace{3mm} \hrule \vspace{4mm}}

\newcommand{\red}[1]{{\color{red} #1}}
\newcommand{\blue}[1]{{\color{blue} #1}}
\newcommand{\mytitle}[1]{{\large{\textbf{#1}}}}
\newcommand{\mysubtitle}[1]{{\normalsize{\textbf{#1}}}}


\usepackage{amsmath}
\DeclareMathOperator*{\argmin}{arg\,min}

\usepackage[normalem]{ulem} %to strike the words
\usepackage{microtype}

% \usepackage{graphicx}
% \usepackage[table,xcdraw]{xcolor}